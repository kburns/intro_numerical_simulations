\documentclass[main.tex]{subfiles}

% Document
\begin{document}

\begin{abstract}

Many of the equations that scientists and engineers use to model the world are simply too complex to be solved analytically.
Instead, researchers today frequently use computers to perform numerical simulations of the systems they're studying.
In this project, we'll develop and run simulations of a wide variety of physical systems.
First, we'll learn about the algorithms that allow a computer to solve very general systems of equations.
We'll then implement and test several of those algorithms ourselves using the Python programming language.
From there, we'll branch out and use these tools to study the physics of a variety of different problems.
Possibilities include simulating the orbits of planets and moons, the true motion of baseballs with air resistance, and the chaotic behavior of swinging pendulums and chains.
A key point of this project is the generality and flexibility of the algorithms we'll be using, so we can easily adapt and expand our models to look at different aspects of these problems, or study entirely different problems that we think of along the way.

\end{abstract}

\end{document}
